% write your main thesis in logical individual chapters
% for better organization use a separate folder for images
\graphicspath{{img/}}









%==============================
\chapter{Pregled metod medplaformnega razvoja}
\label{chap:overview}


%-----
\section{Celovit}

Celovita metoda za razvoj uporablja eno samo okolje, s pomočjo katerega nato aplikacijo izvozimo za ralične platforme - tako imenovani princip ``piši enkrat, uporabljaj povsod'', ki nam je znan iz programskega jezika Java. Zajema tako orodja za grafični vmesnik, kot prenosljivost programske kode.

\subsection{Qt}

Qt\footnote{\href{http://qt-project.org}{http://qt-project.org}} je ogrodje za grafično programiranje za več platform s pomočjo jezika C++ in QML\footnote{Qt Meta Language ali Qt Modeling Language, vir \href{http://en.wikipedia.org/wiki/QML}{Wikipedia}}. Omogoča nam sočasni razvoj za platforme OSX, Linux, Windows, Android in iOS. Podpira tudi uporabo HTML5\footnote{\href{http://en.wikipedia.org/wiki/HTML5}{http://en.wikipedia.org/wiki/HTML5}} namesto QML, kar pomeni, da spletni razvijalci lahko uporabijo že obstoječe znanje in učenje novega jezika ni potrebno.

Qt projekt je povsem odprtokoden in dovoljuje uporabo v skladu z licencama GPL v3\footnote{\href{http://www.gnu.org/copyleft/gpl.html}{http://www.gnu.org/copyleft/gpl.html}} in LGPL v2.1\footnote{\href{https://www.gnu.org/licenses/old-licenses/lgpl-2.1.html}{https://www.gnu.org/licenses/old-licenses/lgpl-2.1.html}}, a če želite orodje uporabiti za razvoj mobilne aplikacije, boste morali za to odšteti 149\$ mesečno.

Glavne slabosti Qt so neskladnost z izgledom ostalih aplikacij na mobilnih platformah, plačljiva licenca za razvoj mobilnih aplikacij ter končna velikost samih programov.

\subsection{Xamarin}

Xamarin\footnote{\href{https://xamarin.com}{https://xamarin.com}} je ogrodje za sočasen razvoj aplikacij za platforme iOS, Android, Mac in Windows v jeziku C\#. Izjaha iz projekta Mono\footnote{\href{http://www.mono-project.com}{http://www.mono-project.com}}, ki omogoča uporabo ogrodja .NET na različnih platformah. Cenovno ni ravno ugoden, saj se cene začnejo pri 299\$/mesec za vsakega razvijalca in vsako platformo.

Ogrodje omogoča razvoj aplikacij, katerih izgled je skladen z ostalimi aplikacijami na izbrani platformi. Kot primer si lahko ogledamo aplikacijo za poslušanje glasbe Rdio\footnote{\href{https://www.rdio.com}{https://www.rdio.com}}, ki je na voljo za iOS, Android in Windows Phone.

Glavna slabost ogrodja Xamarin je cena. Za majhno ekipo je začetna cena enostavno previsoka. Vprašljiva je hitrost dodajanja funckionalnosti posameznih platform, ko se te nadgradijo, določen riziko predstavlja tudi muhavost posameznih platform pri omejitvah uporabe tega ogrodja.

\subsection{Adobe Air}

Adobe Air je brezplačno ogrodje, ki omogoča zagon iste aplikacije na platformah iOS, Android, Mac, Windows in Linux. Čeprav za razvoj namiznih aplikacij omogoča uporabo HTML in Javascript, je za razvoj mobilnih aplikacij omejen na uporabo jezika ActionScript. V času pisanja diplomske naloge ogrodje ne omogoča zagon na platformi Windows Phone.

Kot glavno slabost ogrodja Adobe Air bi navedel upadanje zanimanja za orodje Flash. Špekuliramo lahko tudi o planih podjetja Adobe, saj so pred kratkim kupili podjetje Nitobi, ki je avtor ogrodja PhoneGap (katerega si ga bomo ogledali v nadaljevanju).

%-----
\section{Hibriden}

Hibridna metoda za razvoj aplikacij uporablja spletne tehnologije v sožitju z kodo za posamezno platformo (t.i. premostitvena tehnika), ki omogoča dostop do glavnih funkcij naprav (kot so kamera, pospeškomer in podobno).

\subsection{Apache Cordova / PhoneGap}

Ogrodje Apache Cordova\footnote{\href{http://cordova.apache.org}{http://cordova.apache.org}} je odprtokodni projekt, ki omogoča objavo spletnih aplikacij kot domorodne. V času pisanja diplomske naloge ogrodje podpira iOS, Android, Windows Phone, Blackberry, Palm WebOS, Bada in Symbian. Na vseh omenjenih platformah nam ogrodje Apache Cordova omogoča dostop do funkcij naprave, ko so naprimer kamera in pospeškomer. 

Projekt PhoneGap\footnote{\href{http://phonegap.com}{http://phonegap.com}} je dejansko samo ena od distribucij projekta Apache Cordova, ki poleg vseh obstoječih funkcionalnosti ponuja tudi razne storitve na katerih delajo v podjetju Adobe.

Za razvoj aplikacij razvijalci lahko uporabljajo spletne tehnologije HTML, CSS in JavaScript. S pomočjo ogrodij jQuery Mobile in Sencha Touch je možno izdelati aplikacije, katerih izgled je zelo lep približek ostalim aplikacijam na izbrani platformi. Če naletimo na funkcijo naprave, do katere nimamo dostopa, ali ugotovimo da je JavaScript za določene naloge premalo učinkovit, lahko preprosto spišemo lasten vtičnik, ki služi kot most med JavaScript in domorodno kodo.

Glavna prednost ogrodja Apache Cordova in predvsem distribucije PhoneGap je izredno nizka pregrada. Priporoča se predvsem za izdelavo prototipnih aplikacij, saj nam omogoča hiter razvoj in iteracijo.

Glavna slabost tega pristopa tiči v performanci in odzivnosti aplikacije, saj ta za prikazovanje izkorišča vgrajeno spletno okno. Trenutno je težko izdelati aplikacije, ki so grafično zahtevnejše, kar pomeni še toliko bolj pereč problem na napravah s slabšimi karakteristikami.

\subsection{Appcelerator Titanium}

Ogrodje Titanium nam omogoča izdelavo aplikacij za več platform hkrati s pomočjo JavaScript okolja, ki služi kot abstrakcijska plast med našo aplikacijo in domorodno kodo. Aplikacijo gradimo s pomočjo jezika JavaScript, ki se med uporabo aplikacije izvaja s pomočjo V8 (Android) in JavaScriptCore (iOS) ali vgrajenega JavaScript okolja (če aplikacijo poganjamo v brskalniku). Za pravilen vizualen izgled skrbijo namestniški elementi, ki uporabljajo domorodne grafične elemente, kar pomeni da vizualno aplikacije ne ločimo od ostalih domorodnih aplikacij. V času pisanja diplomske naloge ogrodje podpira iOS, Android, Blackberry, Tizen in spletne aplikacije.

Glavna prednost ogrodja Titanium ni t.i. način piši enkrat, uporabljaj povosd; njegova prednost je da lahko celotno aplikacijo izdelamo v enem jeziku - JavaScript-u. Le redko se bomo srečali z domorodno kodo, saj ogrodje nudi široko paleto knjižnic.

Glavna slabost ogrodja je počasno dodajanje novih platform zaradi obsežnosti dela, ki ga tak podvig zahteva. Določene knjižnice za delo z domorodnimi elementi tudi niso najbolj performančne, manjka pa tudi napovedana podpora platformi Windows Phone.

%-----
\section{Deljen}

Deljena metoda za razvoj aplikacij omogoča uporabo dela aplikacijske kode na vseh platformah za katere razvijamo. To lahko naredimo s pomočjo vgradnega skriptnega jezika (Lua), s pomočjo prevajanja iz izbranega programskega jezika v domorodnega (Haxe, XMLVM, emscripten) ali pa z uporabo programskega jezika C++ in programskimi ovoji, s katerimi pripravimo knjižnico za vgradnjo v druge platforme.

\subsection{Lua}

Lua\footnote{\href{http://www.lua.org}{http://www.lua.org}} je preprost vgrandi skriptni jezik, ki ga odlikuje hitrost izvajanja in procesorska nezahtevnost, kar pomeni, da je kot nalašč za hitro izdelavo prototipov. Vgradimo ga lahko v platforme Android, iOS, Symbian in Windows Phone, z nekaj potrpljenja pa lahko isto kodo zaženemo tudi v spletni aplikaciji.

Čeprav je jezik Lua preprost za uporabo, se izkaže da za kompleksnejše knjižnice ni primeren. Manjka tudi Unicode podpora in boljša podpora rokovanju z napakami.

Standardna knjižnica?

\subsection{Haxe}

Haxe\footnote{\href{http://haxe.org}{http://haxe.org}} zase pravi, da je večplatformski programski jezik. Razvijalec lahko svojo aplikacijo napiše v jeziku Haxe, nato pa jo s pomočjo prevajalnika prevede v JavaScript, C++, C\# ali Javo.

Pri prevajanju aplikacije v ciljni jezik Haxe povzroči ne tako zanemarljivo napihjenje kode.

\subsection{XMLVM}

XMLVM\footnote{\href{http://xmlvm.org}{http://xmlvm.org}} spada v isti razred kot Haxe - tako imenovanih prevajalcev iz enega jezika v drugega (ang. cross-compilers), a se XMLVM tega loti na drugačen način. Medtem ko Haxe prevaja na nivoju izvorne kode, XMLVM to počne na nivoju zlogovne kode (ang. byte code). Izvorna koda je lahko napisana za navidezne stroje (ang. virtual machine) JVM, .NET CLI ali Ruby YARV, medtem ko je rezultat delujoč program za JVM, .NET CLI, Javascript, Pyhton, Objective-C in C++.

// slika JVM, .NET CLI, Ruby YARV -> JVM, .NET CLI, JavaScript, Python, Objc, C++

Projekt izgleda zelo ambiciozen, a vse kaže da je šlo le za akademsko raziskavo, saj je v času pisanja diplome minilo že več ko leto dni odkar se je izvorna koda posodobila. Kljub temu se mi je projekt zdel zanimiv.

\subsection{C++ in emscripten}

V kolikor nobena od naštetih možnosti ne zadošča našim potrebam, želeli pa bi vseeno imeti deljeno knjižnico, obstaja še ena možnost: uporaba jezika C++\footnote{\href{http://www.cplusplus.com}{http://www.cplusplus.com}} in projekta emscripten\footnote{\href{http://emscripten.org}{http://emscripten.org}}.

C++ je eden izmed najbolj razširjenih programksih jezikov. V času pisanja diplomske naloge zaseda četrto mesto na lestvici najbolj popularnih jezikov\footnote{\href{http://www.tiobe.com/index.php/content/paperinfo/tpci/index.html}{http://www.tiobe.com/index.php/content/paperinfo/tpci/index.html}}, pred njim so samo C, Java in Objecitve-C. Uporablja se ga v raznolikih projektih, od prevajalnikov, strežnikov, do video igric.

Emscripten je projekt Mozilinih laboratorijev ki omogoča prevajanje iz LLVM\footnote{Low Level Virtual Machine} zlogovne kode v skriptni jezik JavaScript. LLVM si lahko predstavljamo kot vmesni sloj med izvorno (C, C++, Objective-C, Java, C#) in strojno kodo, ki skrbi poskrbi za visoko optimizacijo vmesne kode, to pa lahko potem prevedemo v ustrezen nabor ukazov za posamezne procesorje (ARM, x86 itd.). Emscripten tako predstavlja zadnjo fazo prevajalnika, le da vmesne kode iz LLVM ne prevede v ukaze specifičnega procesorja, ampak prevede nazaj v jezik JavaScript. To seveda pomeni da lahko prevedemo skoraj vsak program (z določenimi omejitvami) v JavaScript in ga zaženemo v brskalniku. Celo grafično zahtevne aplikacije niso problematične, saj emscripten za prevod v JavaScript uporablja asm.js\footnote{\href{http://asmjs.org}{http://asmjs.org}}, kar je podmnožica jezika JavaScript, ki jo JavaScript pogoni znajo dobro optimizirati.

%==============================
\chapter{Razvoj knjižnice}
\label{chap:development}

%-----
\section{C++}

%-----
\section{Emscripten}

%-----
\section{Omejitve}

%==============================
\chapter{Vključitev knjižnice v različne platforme}
\label{chap:cross-platform}

%-----
\section{iOS}

%-----
\section{Android}

%-----
\section{Windows Phone}

%-----
\section{Spletna aplikacija}
