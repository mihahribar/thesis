% write your main thesis in logical individual chapters
% for better organization use a separate folder for images
\graphicspath{{img/}}









%==============================
\chapter{Pregled metod medplaformnega razvoja}
\label{chap:overview}


%-----
\section{Celovit}

Celovita metoda za razvoj uporablja eno samo okolje, s pomočjo katerega nato aplikacijo izvozimo za ralične platforme. Tako imenovani princip ``piši enkrat, uporabljaj povsod'', ki nam je znan iz programskega jezika Java. Java nažalost iz zgodovinskih razlogov ni na voljo na dveh od ciljnih platform (iOS in Windows Phone) nam preostaneta še dve metodi: Qt in Xamarin.

\subsection{Qt}

\subsection{Xamarin}

\subsection{Adobe Air}

%-----
\section{Hibriden}

\subsection{PhoneGap}

\subsection{Appcelerator Titanium}

%-----
\section{Deljen}

\subsection{Lua}

\subsection{Haxe}

\subsection{XMLVM}

\subsection{C++ in emscripten}

%==============================
\chapter{Razvoj knjižnice}
\label{chap:development}

%-----
\section{C++}

%-----
\section{Emscripten}

%-----
\section{Omejitve}

%==============================
\chapter{Vključitev knjižnice v različne platforme}
\label{chap:cross-platform}

%-----
\section{iOS}

%-----
\section{Android}

%-----
\section{Windows Phone}

%-----
\section{Spletna aplikacija}
