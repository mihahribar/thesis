% write your main thesis in logical individual chapters
% for better organization use a separate folder for images
\graphicspath{{img/}}









%==============================
\chapter{Pregled metod medplaformnega razvoja}
\label{chap:overview}


%-----
\section{Celovit}

Celovita metoda za razvoj uporablja eno samo okolje, s pomočjo katerega nato aplikacijo izvozimo za ralične platforme - tako imenovani princip ``piši enkrat, uporabljaj povsod'', ki nam je znan iz programskega jezika Java. Zajema tako orodja za grafični vmesnik, kot prenosljivost programske kode.

\subsection{Qt}

Qt\footnote{\href{http://qt-project.org}{http://qt-project.org}} je ogrodje za grafično programiranje za več platform s pomočjo jezika C++ in QML\footnote{Qt Meta Language ali Qt Modeling Language, vir \href{http://en.wikipedia.org/wiki/QML}{Wikipedia}}. Omogoča nam sočasni razvoj za platforme OSX, Linux, Windows, Android in iOS. Podpira tudi uporabo HTML5\footnote{\href{http://en.wikipedia.org/wiki/HTML5}{http://en.wikipedia.org/wiki/HTML5}} namesto QML, kar pomeni, da spletni razvijalci lahko uporabijo že obstoječe znanje in učenje novega jezika ni potrebno.

Qt projekt je povsem odprtokoden in dovoljuje uporabo v skladu z licencama GPL v3\footnote{\href{http://www.gnu.org/copyleft/gpl.html}{http://www.gnu.org/copyleft/gpl.html}} in LGPL v2.1\footnote{\href{https://www.gnu.org/licenses/old-licenses/lgpl-2.1.html}{https://www.gnu.org/licenses/old-licenses/lgpl-2.1.html}}, a če želite orodje uporabiti za razvoj mobilne aplikacije, boste morali za to odšteti 149\$ mesečno.

Če bi izpostavili dva projekta, ki uporabljata ogrodje Qt, bi omenil medijski predvajalnik VLC in program Mathematica.

Glavne slabosti Qt so neskladnost z izgledom ostalih aplikacij na mobilnih platformah, plačljiva licenca za razvoj mobilnih aplikacij ter končna velikost samih programov.

\subsection{Xamarin}

Xamarin\footnote{\href{https://xamarin.com}{https://xamarin.com}} je ogrodje za sočasen razvoj aplikacij za platforme iOS, Android, Mac in Windows v jeziku C\#. Izjaha iz projekta Mono\footnote{\href{http://www.mono-project.com}{http://www.mono-project.com}}, ki omogoča uporabo ogrodja .NET na različnih platformah. Cenovno ni ravno ugoden, saj se cene začnejo pri 299\$/mesec za vsakega razvijalca in vsako platformo.

Ogrodje omogoča razvoj aplikacij, katerih izgled je skladen z ostalimi aplikacijami na izbrani platformi. Kot primer si lahko ogledamo aplikacijo za poslušanje glasbe Rdio\footnote{\href{https://www.rdio.com}{https://www.rdio.com}}, ki je na voljo za iOS, Android in Windows Phone.

Glavna slabost ogrodja Xamarin je cena. Za majhno ekipo je začetna cena enostavno previsoka. Vprašljiva je hitrost dodajanja funckionalnosti posameznih platform, ko se te nadgradijo, določen riziko predstavlja tudi muhavost posameznih platform pri omejitvah uporabe tega ogrodja.

\subsection{Adobe Air}

Adobe Air je brezplačno ogrodje, ki omogoča zagon iste aplikacije na platformah iOS, Android, Mac, Windows in Linux. Čeprav za razvoj namiznih aplikacij omogoča uporabo HTML in Javascript, je za razvoj mobilnih aplikacij omejen na uporabo jezika ActionScript. V času pisanja diplomske naloge ogrodje ne omogoča zagon na platformi Windows Phone.

Kot glavno slabost ogrodja Adobe Air bi navedel upadanje zanimanja za orodje Flash. Špekuliramo lahko tudi o planih podjetja Adobe, saj so pred kratkim kupili podjetje Nitobi, ki je avtor ogrodja Phone Gap (katerega si ga bomo ogledali v nadaljevanju).

%-----
\section{Hibriden}

Hibridna metoda za razvoj aplikacij uporablja spletne tehnologije v sožitju z kodo za posamezno platformo (t.i. premostitvena tehnika), ki omogoča dostop do glavnih funkcij naprav (kot so kamera, pospeškomer in podobno).

\subsection{Apache Cordova / PhoneGap}

Ogrodje Apache Cordova\footnote{\href{http://cordova.apache.org}{http://cordova.apache.org}} je odprtokodni projekt, ki vsebuje vtičnike za uporabo funckionalnosti naprav, kot so kamera in pospeškomer. V času pisanja diplomske naloge ogrodje podpira iOS, Android, Windows Phone, Blackberry, Palm WebOS, Bada in Symbian. Projekt PhoneGap\footnote{\href{http://phonegap.com}{http://phonegap.com}} je dejansko samo ena od distribucij projekta Apache Cordova, ki poleg vseh obstoječih funkcionalnosti ponuja tudi razne storitve na katerih delajo v podjetju Adobe.

Za razvoj aplikacij razvijalci lahko uporabljajo spletne tehnologije HTML, CSS in JavaScript. S pomočjo ogrodij jQuery Mobile in Sencha Touch je možno izdelati aplikacije, katerih izgled je zelo lep približek ostalim aplikacijam na izbrani platformi.

Glavna slabost tega pristopa tiči v performanci in odzivnosti aplikacije. Trenutno je tudi težko izdelati aplikacije, ki so grafično zahtevnejše.

\subsection{Appcelerator Titanium}

%-----
\section{Deljen}

\subsection{Lua}

\subsection{Haxe}

\subsection{XMLVM}

\subsection{C++ in emscripten}

%==============================
\chapter{Razvoj knjižnice}
\label{chap:development}

%-----
\section{C++}

%-----
\section{Emscripten}

%-----
\section{Omejitve}

%==============================
\chapter{Vključitev knjižnice v različne platforme}
\label{chap:cross-platform}

%-----
\section{iOS}

%-----
\section{Android}

%-----
\section{Windows Phone}

%-----
\section{Spletna aplikacija}
