








%==============================
\chapter{Ugotovitve}
\label{chap:conclusion}

V diplomski nalogi smo pokazali, kako je možno poenostaviti sočasni razvoj aplikacij za več različnih platform. S pomočjo projekta Emscripten in jezikovnih ovojev smo uspešno uporabili isto C++ knjižnico v iOS, Android, Windows Phone in spletni aplikaciji. Na ta način lahko skoraj celotno logiko aplikacije prenesemo v knjižnico, ki si jo vse platforme med seboj delijo. Še vedno moramo za vsako platformo posebej razviti uporabniški vmesnik, a lahko tako aplikacijo povsem prilagodimo zmožnostim operacijskega sistema.

Naloga se ni izkazala za prav preprosto, kar je bilo tudi pričakovano, predvsem zaradi razlik med izbranimi platformami. Kot smo pokazali, performančno prikazana rešitev končnih aplikacij najbrž ne bo preveč pohitrila, a to vendarle ni bil namen diplomske naloge.

V bližnji prihodnosti lahko pričakujemo, da opisana rešitev ne bo več edini način vključitve knjižnjice v različne platforme. Apple je že začel prvi korak z ogrodjem \texttt{JavaScriptCore}, ki omogoča boljše mešanje JavaScript in domorodne kode. Podobne rešitve lahko kmalu pričakujemo tudi od ostalih izbranih platform, kar bo znatno olajšalo razvoj medplatformnih knjižnjic. Ko pride do tega bo poglavje \ref{chap:javascript} lahko lepo izhodišče za novo raziskavo.

Bralec, ki bi opisane primere rad preizkusil, lahko izvorno kodo najde na spletni strani \href{https://github.com/mihahribar/thesis}{github.com/mihahribar/thesis}.