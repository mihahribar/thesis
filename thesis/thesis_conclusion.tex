








%==============================
\chapter{Ugotovitve}
\label{chap:conclusion}

V diplomski nalogi smo pokazali, kako je možno poenostaviti sočasni razvoj aplikacij za več različnih platform. S pomočjo projekta emscripten in destinacijskih ovojev smo uspešno uporabili isto C++ knjižnico v iOS, Android, Windows Phone in spletni aplikaciji. Naloga se ni izkazala za prav preprosto, kar je bilo tudi pričakovano, predvsem zaradi razlik med izbranimi platformami. Diplomska naloga mi je ponudila priložnost osvežiti svoje znanje jezikov C++, Java, Objective-C, C\# in JavaScript, ki jih v svojem delu ne uporabljam ravno vsakodnevno.

V bližnji prihodnosti pričakujem, da opisana rešitev ne bo več edini način vključitve knjižnjice na različne platforme. Apple je že začel prvi korak z ogrodjem \texttt{JavaScriptCore}, ki omogoča boljše mešanje JavaScript in domorodne kode. Podobne rešitve pričakujem tudi od ostalih izbranih platform, kar bi znatno olajšalo razvoj medplatformnih knjižnjic. Ko pride do tega, bom za svoje lastne projekte rajši izbral razvoj z jezikom JavaScript, ki smo ga predstavili v poglavju \ref{chap:javascript}.

Bralec, ki bi opisane primere rad preizkusil, lahko izvorno kodo najde na spletni strani \href{https://github.com/mihahribar/thesis}{github.com/mihahribar/thesis}.