% write thesis abstracts
% use \begin{abstract}\end{abstract} for abstract in english
% use \begin{povzetek}\end{povzetek} for abstract in slovene









%==============================
\begin{Povzetek}

%-----
Razvoj aplikacij za več različnih platform je težaven. Odpira veliko možnosti za napake, oteži testiranje in odpravljen napak ter skoraj onemogoči sočasno nadgrajevanje aplikacij. Rezultat so dolgotrajni razvojni cikli in počasno dodajanje funkcionalnosti, kar v današnjem startup svetu ni zaželjeno.

Kljub različnosti med posameznimi platformami je ponavadi veliko kode z identično funkcionalnostjo, ki jo je potrebno razviti za vsako platformo posebej. Velikokrat je v podjetju za vsako od platform zadolžen drug razvijalec, še bolj pogosto pa razvoj na različnih platformah ne poteka sočasno. Rešitev iz te zagate je razvoj medplatformne knjižnice.

Cilj diplomske naloge je razvoj knjižnice za \href{http://tools.ietf.org/html/rfc5545#section-3.3.10}{RRULE RFC5545} specifikacije, ki omogoča generiranje ponavljajočih se koledarskih dogodkov in jo je možno uporabiti v spletni, iOS, Android in Windows Phone aplikaciji. Našteli bomo možne pristope, navedli prednosti in slabosti, ter na koncu izbrali najbolj primerno rešitev za implementacijo knjižnice.
\end{Povzetek}









%==============================
\begin{Abstract}

%-----
Provide a concise and meaningful abstract of your work. The description should be one or two pages long and subdivded into paragraphs.

There is no need to provide a list of keywords as these are automatically appended at the end of the abstract.
\end{Abstract}
