%signal path for graphics files
\graphicspath{{img/}}







%==============================




%-----
\section{Motivation}
Give a general overview of the motivations for your research, but bear in mind that the introduction should be interesting. If you bore the reader here, then you are unlikely to revive their interest in the following sections. For the first paragraph or two, tradition permits prose that is less dry than the scientific norm. If want to wax lyrical about your topic, here is the place to do it. Remember to keep the proposal short, the sweet spot is anywhere between 3--5 pages, with the list of references included. A thesis proposal should be written in such a way, that it could serve as the leading chapter of your thesis, the thesis introduction. Try to make the reader want to read the kilogram of A4 that has arrived uninvited on their desk. Go to the library and read several thesis introductions. Did any make you want to read the on? Which ones were boring? 

This section might go through several drafts to make it read well and logically, while keeping it short. For this section, I think that it is a good idea to ask someone who is not a specialist to read it and to comment. Is it an adequate introduction? Is it easy to follow? There is an argument for writing this section---or least making a major revision of it---towards the end of the writing. Your introduction should tell where the research is going, and this may become clearer during the writing. 

%-----
\section{Methodology}
Provide descriptions of the methodologies that will be applied during your research. For any thing that will not been done by you or has already been done in your chosen field provide adequate acknowledgement either by referencing the scientific work \cite{adami:1998, bentley:2002, reynolds:1978, reynolds:1987, lebar_bajec:2005a, lebar_bajec:2005b, heppner:1990} or through the Acknowledgements chapter. The Internet and Google can serve as good starting points for writing a thesis, especially since there can be found many tutorials.\footnote{\href{http://www.phys.unsw.edu.au/~jw/thesis.html}{http://www.phys.unsw.edu.au/\textasciitilde jw/thesis.html}} Make clear the way in which your research will differ from anything that has already been done.

%-----
\section{Scientific Contributions}
Explain the expected scientific contributions that will arise from this research. Be brief, list at most three key points that you feel will arise from your research. Bear in mind that a scientific contribution should be publishable \emph{per se} as a scientific research paper. Preferably in a (Social) Science Citation Index (Expanded) indexed journal.
