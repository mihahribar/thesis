%signal path for graphics files
\graphicspath{{img/}}







%==============================
\chapter{Uvod}
\label{chap:introduction}


%-----
Dandanes uporabljamo več različnih naprav sočasno. V lasti imamo najverjetneje prenosni računalnik, pametni telefon in po možnosti še tablico. Ko najdemo aplikacijo, ki nam je všeč, od te pričakujemo brezhibno delovanje na vseh naših napravah.

To je seveda zelo težko doseči, sploh z majhno ekipo. Dodatno se stvari zakomplicirajo, če so vse te naprave na različnih operacijskih sistemih. Tako imam lahko Windows prenosnik, Android telefon in Apple tablico. Kot razvajen uporabnik pričakujem, da je izbrana aplikacija na voljo na vse naštetih platformah.

Za razvijalca smo ravnokar opisali nočno moro. Da zadovolji potrebe uporabnikov, je primoran razviti isto aplikacijo za vsako od platform. Četudi omejimo razvoj na najbolj priljubljene platforme iOS, Android in Windows Phone, smo ravnokar našteli tri povsem različne tehnologije, tri različne jezike in s tem tri priložnosti za povsem različne težave pri implementaciji naše aplikacije. Veliko truda in energije je potrebno, da so te aplikacije poenotene in da skladno sledijo razvoju novih funckionalnosti.

Izkušen razvijalec bo pri predstavitvi problema takoj pomislil na medplatformni razvoj, ki si ga bomo ogledali v drugem poglavju. Omenili bomo tako imenovane ``celotive'' metode, kot so Qt in Xamarain, ``hibridne'' kot sta recimo PhoneGap in Appcelerator Titanium, ter ``deljene'' metode npr. Lua, Haxe in C++.

V tretjem poglavju si bomo ogledali zakaj smo se odločili za razvoj knjižnice s pomočjo C++, ter jo tudi zgraditi. V četrtem poglavju jo bomo ovili v nekaj ovojev in uspešno uporabili v ObjC (iOS), Javi (Android) in C\# (Windows Phone). Predstavili bomo tudi način kako C++ kodo prevesti neposredno v JavaScript s pomočjo orodja emscripten, ter na koncu knjižnico vključili tudi v spletno aplikacijo.

Pa začnimo.
