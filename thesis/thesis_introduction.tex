%signal path for graphics files
\graphicspath{{img/}}







%==============================
\chapter{Introduction}
\label{chap:introduction}


%-----
\section{Motivation}
Give a general overview of the motivations for this thesis, but bear in mind that the introduction should be interesting. If you bore the reader here, then you are unlikely to revive their interest in the materials and methods section. For the first paragraph or two, tradition permits prose that is less dry than the scientific norm. If want to wax lyrical about your topic, here is the place to do it. Try to make the reader want to read the kilogram of A4 that has arrived uninvited on their desk. Go to the library and read several thesis introductions. Did any make you want to read on? Which ones were boring? 

This section might go through several drafts to make it read well and logically, while keeping it short. For this section, I think that it is a good idea to ask someone who is not a specialist to read it and to comment. Is it an adequate introduction? Is it easy to follow? There is an argument for writing this section---or least making a major revision of it---towards the end of the thesis writing. Your introduction should tell where the thesis is going, and this may become clearer during the writing. 

%-----
\section{Scientific Contributions}
Explain the expected scientific contributions that will arise from this research.

%-----
\section{Methodology}
Provide descriptions of the methodologies that will be applied during this research.

%-----
\section{Thesis overview}
Present a brief overview of the thesis (\ie~what will you deal with in the following chapters).

%-----
\subsection{Notation}
If you employed any specific notation (for the sake of clarity you should) then use this section to provide its description, \eg:
%
\begin{itemize}
\item $\mathbb{N} = \left\{1, 2, 3, \ldots\right\}$ the set of all positive natural numbers.
\end{itemize}
