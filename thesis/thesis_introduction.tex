%signal path for graphics files
\graphicspath{{img/}}







%==============================
\chapter{Uvod}
\label{chap:introduction}


%-----
Dandanes uporabljamo več različnih naprav sočasno. V lasti imamo najverjetneje prenosni računalnik, pametni telefon in po možnosti še tablični računalnik. Ko najdemo aplikacijo, ki nam je všeč, od te pričakujemo brezhibno delovanje na vseh naših napravah.

To je seveda zelo težko doseči, sploh z majhno ekipo. Dodatno se stvari zakomplicirajo, če so vse te naprave na različnih operacijskih sistemih. Tako imam lahko Windows prenosnik, Android telefon in Apple tablico. Kot razvajen uporabnik pričakujem, da je izbrana aplikacija na voljo na vse naštetih platformah in da na vseh platformah deluje identično.

Za razvijalca smo ravnokar opisali nočno moro. Da zadovolji potrebe uporabnikov, je primoran razviti isto aplikacijo za vsako od platform. Četudi omejimo razvoj na najbolj priljubljene platforme iOS, Android in Windows Phone, smo ravnokar našteli tri povsem različne tehnologije, tri različne jezike in s tem tri priložnosti za povsem različne težave pri implementaciji naše aplikacije. Veliko truda in energije je potrebno, da so te aplikacije poenotene in da skladno sledijo razvoju novih funckionalnosti.

Izkušen razvijalec bo pri predstavitvi problema takoj pomislil na medplatformni razvoj, ki si ga bomo ogledali v drugem poglavju. Omenili bomo tako imenovane ``celotive'' metode, kot so Qt\cite{qt} in Xamarin\cite{xamarin}, ``hibridne'' kot sta recimo PhoneGap\cite{phonegap} in Appcelerator Titanium\cite{titanium}, ter ``deljene'' metode npr. Lua\cite{lua}, Haxe\cite{haxe} in C++\cite{cpp}.

V tretjem poglavju si bomo ogledali zakaj smo se odločili za razvoj knjižnice s pomočjo C++, ter jo tudi zgradili. V četrtem poglavju jo bomo ovili v nekaj ovojev in uspešno uporabili v Objective-C (iOS), Javi (Android) in C\# (Windows Phone). Predstavili bomo tudi način, kako lahko C++ kodo prevedemo v JavaScript s pomočjo orodja Emscripten\cite{emscripten}, in knjižnico uporabili tudi v spletni aplikaciji.

Pa začnimo.
