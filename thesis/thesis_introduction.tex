%signal path for graphics files
\graphicspath{{img/}}







%==============================
\chapter{Uvod}
\label{chap:introduction}


%-----
Dandanes uporabljamo več različnih naprav sočasno. V lasti imamo najverjetneje prenosni računalnik, pametni telefon in po možnosti še tablico. Ko najdemo aplikacijo, ki nam je všeč od te pričakujemo brezhibno delovanje na vseh naših napravah.

To je seveda težko doseči, sploh z majhno ekipo. Dodatno se stvari zakomplicirajo, če so vse te naprave na različnih operacijskih sistemih. Tako imam lahko Windows prenosnik, Android telefon in Apple tablico. Kako se lahko rešimo iz zagate, da bomo primorani za vsako od platform celotno funckionalnost spisati od začetka? Kaj če želimo imeti aplikacije kar se da podobne ostalim aplikacijam na isti platformi? Uspešna aplikacija se mora zliti z operacijskim sistemom na katerem gostuje. V tem so trenutno še posebej slabe HTML5 aplikacije, saj so ponavadi izgled le imitirajo.

V drugem poglavju si bomo ogledali metode medplatformnega razvoja. Omenili bomo tako imenovane ``celotive'' metode, kot so Qt in Xamarain, ``hibridne'' kot sta recimo PhoneGap in Appcelerator Titanium, ter ``deljene'' metode npr. Lua, Haxe in C++.

V tretjem poglavju si bomo ogledali kako zgraditi medplatformno knjižnico z uporabo C++, jo nato v četrtem poglavju ovili v nekaj ovojev in uspešno uporabili v ObjC, Javi in C\#. Predstavili pa bomo tudi način kako C++ kodo prevesti neposredno v JavaScript in knjižnico vključili tudi v spletno aplikacijo.
